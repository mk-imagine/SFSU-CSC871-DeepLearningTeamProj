\documentclass[12pt]{article}
\usepackage{mysty}
\geometry{top=1in, bottom=1in, left=1in, right=1in}
\doublespacing
\newcommand{\matr}[1]{\mathbf{#1}}
\graphicspath{ {./images/} }

\title{\textbf{CSC 871 - Deep Learning}\\\Large Team Project Proposal - Style Transfer}
\author{\large Mark Kim, Satvik Verma, Fabian Weiland}
\date{\today}

\begin{document}
\maketitle
\vspace{5mm}
\begin{center}
    Instructor:\\
    Prof. Robert Mateescu
\end{center}
\newpage

\section{Description}
Advancements in machine learning have brought generative AI into the mainstream.
Unfortunately, trying to reproduce these kinds of advanced models with limited
compute resources is infeasible.  Nevertheless, there are ways to utilize models
that may not match the state-of-the-art and still produce amazing results while
learning about the foundations such models are built upon.  In our project, we
attempt to implement neural style transfer using pre-trained models and multiple
approaches to the style loss function.  At the very least, we will examine the
gram-matrix method, but aspire to experiment and compare this method with two
other methods: optimal transport and Vincent loss.  Finally, as a reach goal, we
may attempt mixed style transfer using multiple style images to produce the
generated image.  With this reach goal, it will be interesting to see if we can
produce more generalized results that mimic an artist's style rather than just a
single painting.  Likewise, it will be fascinating to see the results from
mixing multiple varying styles into a generated combination image.
% References
% https://towardsdatascience.com/art-with-ai-turning-photographs-into-artwork-with-neural-style-transfer-8144ece44bed
% https://www.kaggle.com/code/basu369victor/style-transfer-deep-learning-algorithm/notebook
% DIFFERENT METHODS - https://wandb.ai/johnowhitaker/style_loss_showdown/reports/An-Explanation-of-Style-Transfer-With-a-Showdown-of-Different-Techniques--VmlldzozMDIzNjg0
% MIXED -
%   https://towardsdatascience.com/mixed-neural-style-transfer-with-two-style-images-9469b2681b54
%   https://cs231n.stanford.edu/reports/2017/pdfs/405.pdf

\section{Dataset}
The style dataset that we will be working with is a collection of paintings from famous
artists ranging from Monet to Pollack which we source from 
\href{https://www.kaggle.com/datasets/ikarus777/best-artworks-of-all-time}{Kaggle}.
This collection consists of a total of 8446 paintings from 50 different artists.
Because we will be producing a generalized model, we will be experimenting with
multiple artworks, which is why we need a variety of style sources.  As for our
base or content images, we will be using various personal, public domain, and
fully licensed sources.

\section{Approach \& Task Assignment}
We will be using the PyTorch library for our project and approach this problem
incrementally and iteratively. Our primary goal is to implement a simple style
transfer using the gram-matrix method with a single style and content image.
Our secondary goal is to expand on this with the optimal transport and Vincent
loss methods and then analyzing the differences and similarities between the
three.  Our final and aspirational goal will be to implement mixed style
transfer.

As our team has varied expertise, we will split the responsibilities
accordingly.  Satvik Verma has experience in computer vision, CNNs, and image
processing, so he will take lead with researching and implementing the image
pre-processing tasks.  Since Mark Kim has the most experience and knowledge of
Mathematics, he will be responsible for researching, advising, explaining, and
writing about the mathematical underpinnings of the entire process.  Fabian
Weiland has broad experience in all stages of software design, so he will take
lead for software design and implementation. The final report will be a group
effort, with each member contributing according to their respective expertise
and to the editing of the entire document.

\section{Discussion}
Since we are building our project incrementally with the intention of creating a
model that may have the flexibility to use different style loss methods and
mixed style transfer, we will need to employ good software design practices with
extensibility and flexibility in mind.  By doing this, we should be able to
implement our primary goal and then build upon it with very little to no
refactoring as our project progresses.  Similarly, if time does not permit us to
expand the project, we will have the option to abandon the more ambitious
portions of the project without affecting the rest of the project.

This project is generative with no quantitative methods of evaluation, so our
assessment of the project will rely mostly on qualitative analysis and subjective
measures of performance.  Currently, our ideas for quantitative evaluation will
likely be based on how quickly each method converges on a suitable result with
respect to the type of task provided to it.  As we progress, we suspect that we
will learn other ways to evaluate the project.
\end{document}